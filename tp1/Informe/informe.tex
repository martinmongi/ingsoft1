\documentclass[spanish, 10pt,a4paper]{article}
\usepackage[spanish]{babel}
\usepackage[utf8]{inputenc}
\usepackage{textcomp}
\usepackage{hyperref}
\usepackage[pdftex]{graphicx}
\usepackage{epsfig}
\usepackage{amsmath}
\usepackage{hyperref}
\usepackage{amssymb}
\usepackage{color}
\usepackage{graphics}
\usepackage{amsthm}
\usepackage{subcaption}
\usepackage{caratula}
\usepackage{fancyhdr,lastpage}
\usepackage[paper=a4paper, left=1.4cm, right=1.4cm, bottom=1.4cm, top=1.4cm]{geometry}
\usepackage[table]{xcolor} % color en las matrices
\usepackage[font=small,labelfont=bf]{caption} % caption de las figuras en letra mas chica que el texto
\usepackage[ruled,vlined,linesnumbered]{algorithm2e}
\usepackage{listings}
\usepackage{float}
\usepackage{pdfpages}
\usepackage{amsfonts}
\usepackage{upgreek}


\color{black}

%%%PAGE LAYOUT%%%
\topmargin = -1.2cm
\voffset = 0cm
\hoffset = 0em
\textwidth = 48em
\textheight = 164 ex
\oddsidemargin = 0.5 em
\parindent = 2 em
\parskip = 3 pt
\footskip = 7ex
\headheight = 20pt
\pagestyle{fancy}
\lhead{IS1 - Trabajo Pr\'actico 1} % cambia la parte izquierda del encabezado
\renewcommand{\sectionmark}[1]{\markboth{#1}{}} % cambia la parte derecha del encabezado
\rfoot{\thepage}
\cfoot{}
\numberwithin{equation}{section} %sets equation numbers <chapter>.<section>.<subsection>.<index>

\newcommand{\figurewidth}{1\textwidth}

\newcommand{\tuple}[1]{\ensuremath{\left \langle #1 \right \rangle }}
\newcommand{\Ode}[1]{\small{$\mathcal{O}(#1)$}}


%El siguiente paquete permite escribir la caratula facilmente
\hypersetup{
  pdftitle={ IS1 - TP1 },
  colorlinks,
  citecolor=black,
  filecolor=black,
  linkcolor=black,
  urlcolor=black 
}

\materia{Ingeniería de Software I}

\titulo{Trabajo Práctico 1}

\subtitulo{Informe y diagramas.}

\grupo{Grupo 2}

\integrante{De Sousa Bispo, Germán}{359/12}{germandesousa@gmail.com}
\integrante{Fernandez, Esteban}{691/12}{esteban.pmf@gmail.com}
\integrante{Kodelia, Erika Natasha}{767/11}{erikankodelia@gmail.com}
\integrante{Mongi Badia, Martín}{422/13}{martinmongi@gmail.com}
\integrante{Sánchez Cano, Gonzalo}{}{xeneize__86@hotmail.com}
\integrante{Wright, Carolina}{876/12}{wright.carolina@gmail.com}

 
\begin{document}
{ \oddsidemargin = 2em
	\headheight = -20pt
	\maketitle
}
	\tableofcontents
	\newpage
\section{Introducción}
	El Ministro de Gobierno quiere modificar el Sistema Electoral Nacional y para ello propone instalar en las escuelas máquinas emisoras de sufragios. Junto con esta incorporación se deberá modificar el Sistema del Centro de Cómputos Nacional para que pueda operar con las máquinas.

	El formato de la votación no presenta cambios, es decir, al igual que en el sistema de boletas que se venía utilizando se permitirá votar por categorías o votar en blanco. 
	
	Además se busca proveer todos los mecanimos necesarios para asegurar el derecho de voto a todos los Electores. Se considerarán las necesidades de los no videntes y personas con movilidad reducida. 

\subsection{Alcance del software}

	Relación con los usuarios: 
\begin{itemize}
\item El diseño del sistema esta puramente pensado para la inclusión de todos los Electores. Cualquiera, más allá de su condición, podrá ejercer el voto sin problema alguno.
\item El sistema asegura que el voto seca realiza de manera anónima y ningún dato personal del Elector queda registrado en la máquina.
\end{itemize}

	Relación con el hardware:
\begin{itemize}
\item Los usuarios interactúan con el sistema a través de la pantalla táctil. Además se cuenta con auriculares y un sistema de guía por medio de audio para los Electores no videntes.
\item El sistema no interactúa con sistemas externos.
\end{itemize}

	Relación con sistemas preexistentes:
\begin{itemize}	
\item Se utiliza el Correo para enviar el aviso a los Presidentes de mesa.
\end{itemize}

	Veremos a lo largo del trabajo estos ítems en detalle.
	
\newpage
\section{Describiendo el problema}

\subsection{Hipótesis de Dominio}
\begin{itemize}
\item Las escuelas tiene las instalaciones necesarias para el acceso de discapacitados
\item Los presidentes de mesa están capacitados para ayudar durante el voto ante dudas de uso del sistema
\item Hay un Encargado de las máquinas por zona
\item Ante falla en máquina de voto, el Encargado de las máquinas solucionará el problema en un tiempo acotado
\item La cantidad de boletas provistas a cada escuela se corresponde, como mínimo, a la cantidad de Electores
\item Los electores votan en el horario establecido
\item Las escuelas tienen cable de red o WIFI para poder comunicarse con los servidores locales y nacionales
\item El día anterior a la votación se realiza la instalación de las máquinas 
\item En caso de un Elector ciego y sordo, el mismo podrá votar con la asesoría de un tutor o bien presentar un certificado de incapacidad.
\end{itemize}

\subsection{Requerimientos}
\begin{itemize}
\item El Elector puede votar una única vez
\item Para poder, votar el Elector debe aparecer en el Padrón de la escuela
\item El voto se realiza de a un Elector por vez
\item Manejar correctamente la situación en la que el Elector presenta alguna discapacidad o es no vidente
\item El Presidente de mesa y los Fiscales no utilizan la máquina durante la elección
\item La máquina permite realizar el voto si la boleta fue insertada
\item Las máquinas tienen conexión a internet
\item Las máquinas se actualizan el día anterior
\item La batería de las máquinas inicia el día con la carga completa
\item El padrón puede consultarse vía página Web o presencialmente
\item Los resultados de la votación pueden consultarse en cualquier momento
\item El Presidente de Mesa y los fiscales pueden consultar información de la máquina en cualquier momento (cantidad de votos, candidatos, etc.)
\item El Presidente de Mesa es el encargado de dar inicio y cierre a la votación
\item El Encargado de Máquinas las actualiza
\item El Encargado de Máquinas las arregla
\item El Encargado de Máquinas las instala en las escuelas correspondientes
\item Reconocidas universidades auditan el software y hardware	
\end{itemize}

\subsection{Principales alternativas}
\subsubsection{Métodos de autenticación}

	Nuestro sistema de votación electrónico, deberá contar con algún método a través del cual asegurarse que no se emitan sufragios ilegítimos (por ejemplo, que un elector vote dos veces o que vote alguien que no estaba en el padrón). Para esto, en nuestro diagrama de objetivos tenemos un o-refinamiento, en el cual planteamos varias formas de autenticación: 

	\begin{itemize}

		\item El Presidente de Mesa le otorga una boleta al Elector, la cual la Máquina imprime y devuelve al Elector. Luego, este deposita la boleta ya impresa en la urna. Este método tiene la ventaja de ser más económico, ya que la máquina de votación no tiene otros requerimientos de hardware que la impresora. Sin embargo, es imposible para el conteo interno de la máquina comprobar que no haya habido fraude, por lo tanto los conteos provisorios podrían ser inexactos, ya que dependen exclusivamente del conteo por software.
		\item El Presidente de Mesa le otorga una boleta al Elector, el cual la coloca en la Máquina. Luego, esta le pide que se autentique. Dicha autenticación se lleva a cabo de a través de métodos biométricos, a través del cual se verifica la identidad del elector, que pertenezca al padrón de la Mesa y que no haya votado previamente. Si es autenticado, la Máquina le permite sufragiar, imprime la boleta y se la devuelve al Elector. Este método tiene la ventaja de que el fraude es improbable. Sin embargo, tiene el costo agregado de que cada Máquina debe contar con un sensor biométrico del tipo elegido. Hay varios tipos de autenticación biométrica, que varían en seguridad y costo de implementación. Para nuestro método consideraremos el que autentica a través de las huellas dactilares, aprovechando que PJN ya cuenta con esa base de datos, y que provee un nivel aceptable de seguridad y costo.
	\end{itemize}


\section{Escenarios informales}

Agregamos unos casos hipotéticos en los cuales aparecerán los agentes y objetivos de una manera más definida a fin de entender mejor el sistema que planteamos.

\subsection{Consulta del padrón}
	Cualquier ciudadano que le corresponda votar en la siguiente fecha de elecciones puede consultar sus datos y lugar designado por dos medios distintos: de forma presencial en cualquier secretaría electoral o a través de la página web. En caso de que los datos sean incorrectos o sus datos no figuren (la persona no se encuentra en el padrón) deberá hacer la denuncia correspondiente en la secretaría y se lo incorporará a la brevedad.
	
	Allí mismo también se podrá acceder a información de los candidatos.

\subsection{Preparación de los espacios de voto y asignación de autoridades de mesa}
	Las autoridades son avisadas con anticipación que han sido designadas como Presidente de mesa por correo. Dado que es un deber, deberán asistir en el día y horario indicado. Cuando ocurra la votación, ellos son los encargados de dar inicio a las mismas habilitando la máquina correspondiente a su mesa y deshabilitandola cuando finalice.		
	
	Un día antes que comience la votación, los Encargados de las Máquinas las preparan y checkean su correcto funcionamiento. Primero deben actualizar los datos y para ello descargan los candidatos correspondientes. Una vez que se encuentran en correcto estado, son llevadas a las escuelas asignadas como sedes de la elección. Allí, estos Encargados revisan y confirman que todo este funcionando como se espera. 
	
	Siempre hay máquinas extras disponibles que pueden ser utilizadas para reemplazar alguna otra que no este en condiciones óptimas.
	

\subsection{Curso de una votación estándar}
	El Elector averigua el lugar en donde le corresponde votar y el día de la votación concurre a la escuela que le fue asignada. Una vez ahí se presenta y anuncia en la mesa. El Presidente de Mesa y Fiscales comprueban los datos, le entregan una boleta vacía y le permiten dirigirse a la máquina emisora de sufragio. 

	Para poder realizar el voto, el Elector ingresa la boleta única en la máquina y a partir de allí comenzará a elegir los candidatos. Una vez que navegó por las categorías o eligió lista completa podrá  realizar un clic en "finalizar votación". A continuación se imprimirá, en el ticket antes ingresado, los datos de los candidatos elegidos. Con la boleta impresa confirmará que los datos corresponden a los ingresados a través del lector que se encuentra en la misma máquina.

	Con el ticket impreso en mano se presenta nuevamente en la mesa y lo ingresa en la urna.	
	
\subsection{Incorrecta elección de candidatos}
	Siguiendo el ejemplo anterior, el Elector realiza la misma votación. Una vez terminada, decide confirmar su voto haciendo uso del lector. Allí se da cuenta que los datos que el ingresó no son lo que quería, es decir, su intención de voto no se ve reflejada en el ticket. Luego, se acerca a la mesa y explica lo sucedido. El Presidente de Mesa rompe el ticket con los datos incorrectos y le entrega uno nuevo. El Elector vuelve a repetir el proceso y finaliza como el caso anterior. 

\subsection{Incorrecto funcionamiento de la máquina}
	Durante el proceso de votación al Elector se le presenta un problema: la máquina no funciona correctamente. Luego de dar conocimiento de lo sucedido al Presidente de Mesa, éste último se comunicará con el Encargado de Máquinas quien concurrirá en un tiempo razonable a reparar la misma. 
	
	Mientras y hasta que se solucione el inconveniente, la mesa se moverá para hacer uso de las máquinas extras que se encuentran en la planta baja  (mismas en las que votarán los Electores con discapacidad). En las mismas,  el Presidente de Mesa seleccionará la mesa que corresponde y el Elector podrá votar con normalidad.

\subsection{Ubicación de los candidatos en la pantalla de la maquina de sufragios}
Por otro lado tenemos 3 posibilidades para la implementación sobre la ubicación de los partidos/candidatos en la pantalla. Para todos los casos la ubicación de los candidatos/partidos es aleatoria para evitar que se pueda deducir a quien está votando el elector:

\begin{itemize}

\item Primero seleccionar el partido que se desea votar y luego al candidato (los tamaños son iguales para todos). Esta opción suele perjudicar a aquellos candidatos cuyos nombres son más conocidos que el partido al que pertenecen. Debido a esto, un elector ante esta situación no sabe que partido seleccionar para poder encontrar al candidato que decidio elegir. Por esta razón, si se quiere ofrecer que sea parejo para todos los candidatos, esta modalidad no es la mejor.

\item La segunda opción que ofrecemos es la totalidad de los candidatos en pantalla, aumentando el tamaño cuanto menor es la popularidad del candidato (en caso de empate, se repartirán con la aleatoridad no se beneficiará más a uno que a otro). Esta modalidad resultar ser más beneficiosa para aquellos candidatos que son menos conocidos y/o populares. Por lo tanto esta opción  brinda una igualdad relativa, pero probablemente no preferida por los partidos mas populares, para todos los candidatos (a pesar de mostrarse todos en pantalla).

\item La última opción que ofrecemos es la totalidad de los candidatos en pantalla, con igual tamaño (en caso de ser varios, se repartirán en más de una pantalla pero con la aleatoridad no se beneficiará más a uno que a otro). Esta modalidad resulta ser la mas beneficiosa, entre las presentadas, ya que al tamaño al ser igual para todos y además la totalidad de los candidatos esta en pantalla al mismo tiempo, no perjudica más a uno que a otro.

\end{itemize}

\subsection{Elector discapacitado}
	Como las escuelas cuentan con las instalaciones necesarias para el acceso de cualquier persona, el Elector con discapacidad puede ingresar al establecimiento sin problemas. Una vez allí pueden ocurrir dos escenarios: su mesa se encuentra a su alcance (esta en planta baja) o no.
	
	Si se encuentra en el primer caso, realiza el voto de manera estándar sin complicaciones. 
	
	Si se encuentra en el segundo se avisará en la mesa que le corresponde y el Presidente bajará con la urna hacia el cuarto especialmente preparado para esta situación.	Este cuarto se encuentra en planta baja por lo que el Elector podrá acceder cómodamente. Antes de que el Elector pueda acercase a la máquina, el Presidente de Mesa ingresará el número de mesa correspondiente en la misma. Esto es necesario dado que el conteo de votos se realiza primero por mesa, luego por escuela y por último en la totalidad. Entonces, como este cuarto es para uso de discapacitados que pueden pertenecer a diferentes mesas, la máquina que allí se encuentra sumará votos de distintas mesas si no se indica lo contrario. A continuación el Elector procede a votar normalmente.

\subsection{Elector no vidente}

	En este caso, cuando el Elector se presenta en la mesa se le entregan unos auriculares. Éstos servirán para guiarlo durante el proceso de votación indicando los movimientos a realizar y la ubicación de los candidatos en la pantalla.
	
	Para que pueda comenzar a elegir los candidatos, el Presidente de Mesa se dirige a la máquina, activa el modo audio y conecta los auriculares. Ahora sí, el Elector puede comenzar. Todo el proceso es acompañado por una grabación que indica en que posición de la pantalla se encuentran los distintos candidatos.
	
\subsection{Presidente de Mesa o Fiscal perciben algo extraño}
	Durante el transcurso del día pueden ocurrir distintas situaciones que presten a duda sobre el funcionamiento de las máquinas. Por ejemplo, un Elector mientras realiza el voto hace maniobras extrañas. En tal caso, tanto los Fiscales como el Presidente de Mesa, pueden interferir luego de que el Elector finalice el voto y consultar en la máquina los candidatos ingresados, la cantidad de votos hasta el momento, resultados parciales y/o finales una vez que se haya cerrado la votación. De esta manera se asegura que no a ocurrido una manipulación de los datos. 
	
	Para realizar esta consulta deberán ingresar en la máquina sus boletas especiales que permiten acceso Root. Una vez validada la boleta podrán acceder a estas funciones ocultas y realizar la consulta deseada. 

\subsection{Resultados parciales}
	Gracias a este nuevo sistema, los resultados se van conociendo durante el transcurso del día de elecciones. Éstos son públicos, es decir, cualquier persona que quiera verlos podrá hacerlo a través de la web en cualquier momento.
	Antes de las dos horas, luego del cierre de mesas, se conocerá el resultado final, que también podrá consultarse mediante la web. 

\section{Vistas}
	Para lograr mayor compresibilidad, los diagramas, tanto de contexto como de objetivos se presentarán en formato digital de forma separada. A la hora de la entrega escrita, se tendrán los diagramas impresos. 
	A continuación se harán las aclaraciones pertinentes del diagrama de objetivos:

\subsection{Notas del Diagrama de Objetivos}
\begin{itemize}
	\item En el requerimiento, \textbf{Lograr que Página web solicite información sobre los candidatos a la CCN} : terminamos el refinamiento en ese mismo nodo y dejamos la responsabilidad a la Pagina Web unicamente, aun sabiendo que es un requerimiento previo y necesario que la CCN tenga cargados los candidatos para ofrecerlos a la página web, porque ya lo indicamos en otro sitio del diagrama a la carga de dichos candidatos en la CCN y nos 
parecio relevante volver a mencionar dicho requerimiento.

	\item  La secretaria electoral de cada provincia brindara la informacion referida al padron y a los candidatos.

	\item En el requerimiento, \textbf{Lograr que Página web solicite información sobre los candidatos a la CCN} : terminamos el refinamiento en ese mismo nodo y dejamos la responsabilidad a la Pagina Web unicamente, aun sabiendo que es un requerimiento previo y necesario que la CCN tenga cargados los candidatos para ofrecerlos a la página web, porque ya lo indicamos en otro sitio del diagrama a la carga de dichos candidatos en la CCN y nos parecio relevante volver a mencionar dicho requerimiento.
	\end{itemize}
\newpage
\section{Conclusión}
	En este trabajo práctico tuvimos un primer acercamiento a un problema desconocido para el grupo. A partir de los distintos diagramas presentados pudimos atacar el pedido de un cliente desde varios puntos distintos a fin de satisfacer las expectativas a medida que ibamos profundizando el conocimiento acerca del problema. También pudimos ver cómo son los primeros pasos para la construcción de un modelo de software.
	
	Tanto del diagrama de objetivos como el de contexto, sirvieron para establecer una idea más clara de cual debería ser la forma de funcionar del proyecto sin enfocarnos solamente en lo que, en última medida será lo que deberemos implementar, si no a mayor escala como solución a un proyecto. Pudimos ver la relación entre los mismos y destacar distintos conceptos en cada uno.

	Con los casos hipotéticos tratamos de cubrir los distintos escenarios, mostrando nuestras ideas y objetivos a cumplir con los agentes que intervienen en cada una.
	
	Por otro lado, también nos acercó a la comunicación con el cliente (en nuestro caso, el tutor) para mitigar dudas y ambiguedades. Se relaciona directamente con el mundo laboral, como parte de metodologías ágiles o desde el trabajo de analista funcional, donde la comunicación con el cliente y entender el problema resulta parte esencial del trabajo. Es por esto que consideramos beneficiosa la utilización de estas metodologías.

	%~ \newpage
	%~ \bibliographystyle{plain}
	%~ \clearpage
	%~ \bibliography{bibliography}
	%~ \addcontentsline{toc}{section}{Referencias}

\end{document}
