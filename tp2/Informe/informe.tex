\documentclass[spanish, 10pt,a4paper]{article}
\usepackage[spanish]{babel}
\usepackage[utf8]{inputenc}
\usepackage{textcomp}
\usepackage{hyperref}
\usepackage[pdftex]{graphicx}
\usepackage{epsfig}
\usepackage{amsmath}
\usepackage{hyperref}
\usepackage{amssymb}
\usepackage{color}
\usepackage{graphics}
\usepackage{amsthm}
\usepackage{subcaption}
\usepackage{caratula}
\usepackage{fancyhdr,lastpage}
\usepackage[paper=a4paper, left=1.4cm, right=1.4cm, bottom=1.4cm, top=1.4cm]{geometry}
\usepackage[table]{xcolor} % color en las matrices
\usepackage[font=small,labelfont=bf]{caption} % caption de las figuras en letra mas chica que el texto
\usepackage[ruled,vlined,linesnumbered]{algorithm2e}
\usepackage{listings}
\usepackage{float}
\usepackage{pdfpages}
\usepackage{amsfonts}
\usepackage{upgreek}


\color{black}

%%%PAGE LAYOUT%%%
\topmargin = -1.2cm
\voffset = 0cm
\hoffset = 0em
\textwidth = 48em
\textheight = 164 ex
\oddsidemargin = 0.5 em
\parindent = 2 em
\parskip = 3 pt
\footskip = 7ex
\headheight = 20pt
\pagestyle{fancy}
\lhead{IS1 - Trabajo Pr\'actico 1} % cambia la parte izquierda del encabezado
\renewcommand{\sectionmark}[1]{\markboth{#1}{}} % cambia la parte derecha del encabezado
\rfoot{\thepage}
\cfoot{}
\numberwithin{equation}{section} %sets equation numbers <chapter>.<section>.<subsection>.<index>

\newcommand{\figurewidth}{1\textwidth}

\newcommand{\tuple}[1]{\ensuremath{\left \langle #1 \right \rangle }}
\newcommand{\Ode}[1]{\small{$\mathcal{O}(#1)$}}


%El siguiente paquete permite escribir la caratula facilmente
\hypersetup{
  pdftitle={ IS1 - TP1 },
  colorlinks,
  citecolor=black,
  filecolor=black,
  linkcolor=black,
  urlcolor=black 
}

\materia{Ingeniería de Software I}

\titulo{Trabajo Práctico 1}

\subtitulo{Informe y diagramas.}

\grupo{Grupo 2}

\integrante{De Sousa Bispo, Germán}{359/12}{germandesousa@gmail.com}
\integrante{Fernandez, Esteban}{691/12}{esteban.pmf@gmail.com}
\integrante{Kodelia, Erika Natasha}{767/11}{erikankodelia@gmail.com}
\integrante{Mongi Badia, Martín}{422/13}{martinmongi@gmail.com}
\integrante{Sánchez Cano, Gonzalo}{}{xeneize__86@hotmail.com}
\integrante{Wright, Carolina}{876/12}{wright.carolina@gmail.com}

 
\begin{document}
{ \oddsidemargin = 2em
	\headheight = -20pt
	\maketitle
}
	\tableofcontents
	\newpage
\section{Introducción}
	El Ministro de Gobierno quiere modificar el Sistema Electoral Nacional y para ello propone instalar en las escuelas máquinas emisoras de sufragios. Junto con esta incorporación se deberá modificar el Sistema del Centro de Cómputos Nacional para que pueda operar con las máquinas.

	El formato de la votación no presenta cambios, es decir, al igual que en el sistema de boletas que se venía utilizando se permitirá votar por categorías o votar en blanco. 
	
	Además se busca proveer todos los mecanimos necesarios para asegurar el derecho de voto a todos los Electores. Se considerarán las necesidades de los no videntes y personas con movilidad reducida. 
	
\section{Presunciones}

\section{Vistas}
	A continuación se visualizarán los diagramas realizados junto con la explicación necesaria para cada tipo:
\subsection{Diagrama de Contexto}
	Se agrega el diagrama de contexto entregado en el trabajo práctico anterior con las modificaciones necesarias para adptarse a los O-refinamientos elegidos. Esto corresponde a no realizar una autenticación por parte del Elector a la máquina de sufragio a la hora de realizar la votación (alcanza con insertar la boleta). Además, la habilitación del modo audio de la máquina de sufragio por parte del Presidente de Mesa se realiza solamente a través de la inserción de los auriculares. 
\par El diagrama de contexto resultante es el siguiente:

\vspace{\baselineskip}
    \begin{center}
                \includegraphics[scale=0.40]{imagenes/contexto/DiagramaDeContextoTP2.png}
                \\
                \vspace{1pt}
                \footnotesize\textit{}
        \end{center}
\vspace{\baselineskip}


\subsection{Diagrama de Caso de Uso}

\subsection{Diagrama de Clases}
\subsubsection{OCL}
\begin{itemize}
	\item Si el \textit{candidato} se postula para algún cargo de la \textit{elección}
\\	\textbf{Context: }  Elección
\\	\textbf{def: }PerteneceACandidatosDeElección(candidatoABuscar: Candidato, elección: Elección) : bool = elección.se votan $\rightarrow$ select(cargo \textbar cargo.postulaciones$\rightarrow$select(candidato \textbar candidato.Dni = candidatoABuscar.Dni).size() =1).size()=1

	\item Cada elector de una elección pertenece al padrón de la elección.
\\	\textbf{Context: }  Elector
\\	\textbf{inv: }self.participa en $\rightarrow$ forAll(eleccion \textbar eleccion.padron.electores$\rightarrow$select(elector \textbar self.Dni = elector.Dni).size() = 1)

	\item Los candidatos seleccionados en la boleta deben ser candidatos de la elección.
\\	\textbf{Context: }  Boleta
\\	\textbf{inv: }self.candidatos$\rightarrow$ forAll(candidatoEnBoleta \textbar self.PerteneceACandidatosDeElección(candidatoEnBoleta, self.elección))

	\item Los barrios asignados al encargado de máquinas pertenecen a una misma ciudad.
\\	\textbf{Context: }  Encargado de Máquina
\\	\textbf{inv: } self.asignados$\rightarrow$forAll(barrio1, barrio2 \textbar barrio1.Ciudad = barrio2.Ciudad)
\end{itemize}

\subsection{Diagrama de Actividad}

\subsection{FSM}

\section{Discusión}
	
\section{Conclusión}

	%~ \newpage
	%~ \bibliographystyle{plain}
	%~ \clearpage
	%~ \bibliography{bibliography}
	%~ \addcontentsline{toc}{section}{Referencias}

\end{document}
